%% Introduction %%%%%%%%%%%%%%%%%%%%%%%%%%%%%%%%%%%%%%%%%%%%%%%%%%%%%%%%
\section{Introduction}
\label{sec:intro}

% What is essential to say?

In this paper, we present a decision procedure for solving systems of
integer linear constraints where each expression is subject
to both upper and lower bounds.  Such systems have the form:
%
\begin{equation}
    \label{eq:prob-inequalities}
    \begin{array}{ccccc}
        l_1    & \le & a_{11} x_1 + \cdots + a_{1n} x_n & \le & u_1 \\
        l_2    & \le & a_{21} x_1 + \cdots + a_{2n} x_n & \le & u_2 \\
        \vdots &     & \vdots                         &     & \vdots \\
        l_m    & \le & a_{m1} x_1 + \cdots + a_{mn} x_n & \le & u_m
    \end{array}
\end{equation}
%
where $l_i, u_i, a_{ij}$ are rational constants and the $x_i$ are unknown
integer variables.
%
As a more compact notation, we use $\v{l} \le \mat{A} \v{x} \le \v{u}$
to denote systems with this form.


% The problem is also
%known as \emph{integer linear programming} (ILP), though we are only concerned
%with the feasibility question here.

Our decision procedure is based on a variant of the Schnorr-Euchner
algorithm~\cite{Schnorr-Euchner} for computing the closest lattice
element to a target point, and is implemented in a tool which we call
\emph{BLT}.  The decision procedure reduces the constraint problem to
the problem of checking whether there is a common point $\v{y}$ in
both the lattice $\mathcal{L}_{\mat{A}}$ generated by the columns of $\mat{A}$
and the hyperrectangle containing the points between $\v{l}$ and
$\v{u}$.

%BLT does this by
%rescaling the coefficients in $\v{l}$, $\v{u}$ and $\mat{A}$
%attempting to find the lattice point closest (with respect to the $\linf{}$-norm) to
%the point $p$ in the center of the hyperrectangle.

%The main difference is the algorithm uses the $L^\infty$-norm to compute distances while %Schnorr-Euchner uses the $L^2$-norm.  BLT first
%rescales the
% preprocesing the formula with the form~\eqref{eq:prob-inequalities}, BLT computes the center %point $P$ of the hyperrectangle containing the points between $L$ and $U$.  To check %satisfiability, BLT attempts to find an assignment to $\v{x} \in \ZZ^n$ such that $A\v{x}$ is
%the closest point to $\v{P}$ in the lattice generated by the columns of $A$.
%If during the search BLT can find a point $A \v{x}$ inside the hyperrectangle defined
%by $L$ and $U$, then BLT returns SAT with the coefficients $\v{x}$.  If it can prove
%no such point exists, then it returns UNSAT.

We developed BLT while trying to apply constraint solving to signal
processing algorithms.  In particular, we were studying the problem of
\emph{reversing JPEG decompression}, that is finding JPEGs that decompress
into images that satisfy given constraints.  This could be used to encode
specific values on pixels in the image, perhaps for steganographic purposes.
As we will show, this problem can be expressed as an integer linear constraint
problem with the form~\eqref{eq:prob-inequalities} over $64$ variables.

Before developing BLT, we had generated various instances of this
problem, including both unsatisfiable and satisfiable cases.  We
applied several SMT solvers, including Yices~\cite{Dutertre:cav2014},
CVC4~\cite{DBLP:conf/cav/BarrettCDHJKRT11}, and
Z3~\cite{DeMoura:2008:ZES:1792734.1792766}, as well as an evaluation
version of Gurobi\footnote{Available
  at~\url{http://www.gurobi.com/}.}, an industrial linear programming
solver. The solvers we tried were incapable of solving
all but the most trivial instances of this problem without giving
additional hints. This was true even when after running some of the
problems for months on selected solvers.  In contrast, BLT is able to
solve the majority of the problems in under a second.

%For problems of this shape, we have found that our algorithm is able to solve problems in %seconds that are intractable to all other solvers that we have tried, including %
%  , but is non-trivial to solve; JPEG is a lossy compression algorithm.  Even if a set of %constraints is satisfiable on arbitrary images, it may not be accessible on images obtained %from decompressed images.

%The rest of the paper is organized as follows. First, we describe the lattice
%operations we use in Section~\ref{sec:preliminaries}. Then, we describe the
%algorithm in Section~\ref{sec:dp} and the JPEG preimage problem
%and performance of BLT on that problem in~\ref{sec:jpeg}.  Finally, we
%conclude with a brief discussion of related and future work in
%Section~\ref{sec:final}.