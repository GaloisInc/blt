\section{Related and Future Work}
\label{sec:final}

\textbf{Related Work}.  Our work builds upon the well-known Schnorr-Euchner
algorithm for solving the closest vector problem.  Prior to developing BLT, we
used the closest-vector solver in
\texttt{fplll}~\cite{DBLP:conf/codcry/HanrotPS11} to solve problems.  It uses
the $L^2$ norm, and thus is not a decision procedure.  However, we found that
it could often find satisfying assignments at quality levels greater than $60$
despite the lack of completeness.

\texttt{LattE} is a program for enumerating lattice points in a rational
polytope \cite{DeLoera2004}. This is asking strictly more than the question
we've discussed. \texttt{LattE} is competitive with commercial branch-and-bound
solvers, the same solvers which perform poorly on the DCT problems in section
\ref{sec:jpeg}. We attempted to evaluate \texttt{LattE} on our DCT problems,
but found that it would crash given a 16 GB memory limit.

\textbf{Future Development}.
We are still working on developing BLT, and have plans to continue
testing it on a wider variety of challenge problems.  We plan to
enable the $\linf$ distance calculation to use exact arithmetic.  We
also plan to explore ways to solve problems with one-sided bounds via
heuristics, and ways to infer unsatisfiable subsets of constraints so
that BLT can be integrated into an SMT solver.  Finally, we would like
to integrate techniques from SAT community such as
conflict-driven-clause learning into the ILP search performed by BLT.
The later should help improve its performance on hard problem instances.

More broadly, it also seems interesting to explore how the calculus and
heuristics that BLT uses can be integrated into other ILP solvers, such as those
based on cutting planes (e.g.~\cite{DBLP:journals/jar/JovanovicM13}).  Those algorithms
are able to work on more general problems as they do not require explicit upper
and lower bounds for all linear expressions.  On the other hand, BLT appears
to be more effective at making effective decisions within the search.

\textit{Acknowledgements}. The authors would like to thank Dejan Jovanovi\'{c},
Grant Passmore, and the reviewers for comments that helped improve this paper.
