%% BOUNDED CONSTRAINT PROBS %%%%%%%%%%%%%%%%%%%%%%%%%%%%%%%%%%%%%%%%%%%%
\section{Preliminaries}
\label{sec:preliminaries}

\newcommand{\indices}[1]{\{\,1\ldots,#1\}}

Throughout this section we use bold capital letter symbols such as
$\mat{A}$ to denote matrices,
bold lower case letters such as $\v{x}$ to denote vectors,
and undecorated lower case letters to denote scalars.  The components of a
vector $\v{x} \in \RR^n$ are denoted by $\v{x}_i$ with $i \in \{1,\ldots,n\}$.
%
We use Greek letters such as $\theta$ to denote functions that assign values
to only some of the coordinates (e.g.,~the function $\theta : Y \to \ZZ$ with
$Y \subseteq \indices{n}$ only assigns values to indices in $Y$).
%
A vectors $\v{x}$ is then just an assignment where all the indices have
been assigned.

\paragraph{Lattices.} A \emph{lattice} is a discrete additive subgroup of a
Euclidean space. A subset $\La \subset \RR^n$ is a lattice if and only if
there is a collection of linearly independent vectors $\v{v}_1, \ldots,
\v{v}_n \in \RR^n$ such that
$\La = \left\{ \sum_{i=1}^n c_i \v{v}_i \, \mid \, c_i \in \ZZ \right\}$.
%
Such a collection is called a \emph{basis} of
$\La$ and the number $n$ is called the \emph{rank}.
%
A lattice generally has many different bases.
%, but when a basis
%$\{\v{v}_1, \ldots, \v{v}_n\}$ is fixed, each point $\v{x} \in \La$
%can be decomposed into a weighted integral
%sum $\v{x} = \Sum_{i=1}^n \v{v}_i \v{u}_i$ of the vectors.
%
\emph{Lattice reduction} is a term used to describe
algorithms for producing a basis that is ``short'' and ``nearly orthogonal'',
a property that is extremely useful in practice~\cite{Lenstra}.

%JHx
Given a fixed basis $\{\v{v}_1, \ldots, \v{v}_n\}$ of $\La$ and a partial
assignment $\theta : Y \to \ZZ$ with $Y \subseteq \indices{n}$,
we define the set of lattice elements $\La_\theta \subseteq \La$ as follows:
%
\begin{equation}
     \label{eq:lat-sublayer}
     \La_\theta := \left\{ \sum_{i=1}^n c_i \v{v}_i \mid c_i \in \ZZ, \,
         i \in \fn{dom}(\theta) \Rightarrow c_i = \theta(i) \right\}.
\end{equation}
%
We call $\La_\theta$ a \emph{sublayer} of $\La$.  The set $\La_\theta$ should be thought
of as the subset of lattice elements which remain after a partial assignment
of coefficients is made.

Our algorithm for finding \emph{integer} solutions will rely on an
underlying solver for systems of \emph{real} solutions.  To model this,
we define the real-affine linear space $\La^\RR_\theta$ containing
$\La_\theta$:
%
\begin{equation}
     \La^\RR_\theta := \left\{ \sum_{i=1}^n c_i \v{v}_i \mid c_i \in \RR, \,
        i \in \fn{dom}(\theta) \Rightarrow c_i = \theta(i) \right\}.
\end{equation}
%
A fundamental problem in the theory of lattices is the \emph{closest vector
problem} (CVP)~\cite{Lenstra}. We introduce it here because it is very
closely related to our approach for solving bounded ILP. CVP has the following
form: given a lattice $\La \subset \RR^n$ and a point $\v{q} \in \RR^n$, find
a vector $\v{z} \in \La$ that is \emph{closest} to $\v{q}$; i.e. a vector for
which $\norm{\v{z}-\v{q}}$ is minimal.
%
Such a closest vector must exist, but it may be difficult to compute and may
not be unique.
%
The problem of deciding whether such a vector exists within a given bound
is known to be NP-hard~\cite{Boas}, though polynomial-time algorithms are
known if the rank of $\La$ is fixed~\cite{Schnorr-hierarchy}.
%
\paragraph{$\linf$ Metric.} In our decision procedure, we work in $\RR^n$
with a different metric than the usual Euclidean metric. The $\linf$ norm is
defined by $\norminf{\v{x}} := \max \{ \abs{x_i} \mid i=1,\ldots,n \}$. It
determines a metric via $d_\infty(\v{x}, \v{y}) := \norminf{\v{x}-\v{y}}$.

With $\linf$, the set of points whose distance from a given point $\v{p}$
is at most $r$ (i.e. the closed ball of radius $r$ around $\v{p}$) is a
hypercube (the $n$-dimensional analogue of a square) each of whose faces
is orthogonal to a coordinate axis. Explicitly,
%
\begin{equation}
    \label{eq:linf-sphere}
    \left\{ \v{x} \in \RR^n \mid d_\infty(\v{x}, \v{p}) \le r \right\} =
    \left\{ (x_1, \ldots, x_n) \mid p_i-r \le x_i \le p_i+r \right\}.
\end{equation}
%
The distance metric can be extended to subsets of $\RR^n$ by taking it to be the
minimum distance between any two points in the subsets, i.e.,
for $X, Y \subseteq \RR^n$,
\begin{equation}
 d_\infty(X, Y) := \min_{(\v{x},\v{y}) \in X{}\times{}Y} d_\infty(\v{x},\v{y})
\end{equation}
%
We remark that when the sets $X$ and $Y$ can be defined by systems of linear
equality and inequality constraints over rational coefficients, then the
$\linf$ distance can be calculated using linear programming techniques by
observing that:
%
\begin{align}
d_\infty(X, Y)
    &= \min \left\{\ d_\infty(\v{x}, \v{y})   \mid \v{x} \in X,\, \v{y} \in Y\ \right\} \\
    &= \min \left\{\ \max_i\{\abs{x_i-y_i}\}  \mid \v{x} \in X,\, \v{y} \in Y\ \right\} \\
    &= \min \left\{\ t \mid \abs{x_i-y_i} \le t,\, \v{x} \in X,\, \v{y} \in Y\ \right\} \\
    &= \min \left\{\ t \mid x_i-y_i \le t,\, -(x_i-y_i) \le t,\, \v{x} \in X\,
      \v{y} \in Y\ \right\}
\label{eq:set_distance}
\end{align}
%
The last line is a real linear optimization problem over the free
variables in the systems of equations used to define $X$ and $Y$.